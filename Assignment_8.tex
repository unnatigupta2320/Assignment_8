\documentclass[journal,12pt,twocolumn]{IEEEtran}

\usepackage{setspace}
\usepackage{gensymb}

\singlespacing


\usepackage[cmex10]{amsmath}

\usepackage{amsthm}

\usepackage{mathrsfs}
\usepackage{txfonts}
\usepackage{stfloats}
\usepackage{bm}
\usepackage{cite}
\usepackage{cases}
\usepackage{subfig}

\usepackage{longtable}
\usepackage{multirow}

\usepackage{enumitem}
\usepackage{mathtools}
\usepackage{steinmetz}
\usepackage{tikz}
\usepackage{circuitikz}
\usepackage{verbatim}
\usepackage{tfrupee}
\usepackage[breaklinks=true]{hyperref}
\usepackage{graphicx}
\usepackage{tkz-euclide}
\usepackage{float}

\usetikzlibrary{calc,math}
\usepackage{listings}
    \usepackage{color}                                            %%
    \usepackage{array}                                            %%
    \usepackage{longtable}                                        %%
    \usepackage{calc}                                             %%
    \usepackage{multirow}                                         %%
    \usepackage{hhline}                                           %%
    \usepackage{ifthen}                                           %%
    \usepackage{lscape}     
\usepackage{multicol}
\usepackage{chngcntr}

\DeclareMathOperator*{\Res}{Res}

\renewcommand\thesection{\arabic{section}}
\renewcommand\thesubsection{\thesection.\arabic{subsection}}
\renewcommand\thesubsubsection{\thesubsection.\arabic{subsubsection}}

\renewcommand\thesectiondis{\arabic{section}}
\renewcommand\thesubsectiondis{\thesectiondis.\arabic{subsection}}
\renewcommand\thesubsubsectiondis{\thesubsectiondis.\arabic{subsubsection}}


\hyphenation{op-tical net-works semi-conduc-tor}
\def\inputGnumericTable{}                                 %%

\lstset{
%language=C,
frame=single, 
breaklines=true,
columns=fullflexible
}
\begin{document}


\newtheorem{theorem}{Theorem}[section]
\newtheorem{problem}{Problem}
\newtheorem{proposition}{Proposition}[section]
\newtheorem{lemma}{Lemma}[section]
\newtheorem{corollary}[theorem]{Corollary}
\newtheorem{example}{Example}[section]
\newtheorem{definition}[problem]{Definition}

\newcommand{\BEQA}{\begin{eqnarray}}
\newcommand{\EEQA}{\end{eqnarray}}
\newcommand{\define}{\stackrel{\triangle}{=}}
\bibliographystyle{IEEEtran}
\providecommand{\mbf}{\mathbf}
\providecommand{\pr}[1]{\ensuremath{\Pr\left(#1\right)}}
\providecommand{\qfunc}[1]{\ensuremath{Q\left(#1\right)}}
\providecommand{\sbrak}[1]{\ensuremath{{}\left[#1\right]}}
\providecommand{\lsbrak}[1]{\ensuremath{{}\left[#1\right.}}
\providecommand{\rsbrak}[1]{\ensuremath{{}\left.#1\right]}}
\providecommand{\brak}[1]{\ensuremath{\left(#1\right)}}
\providecommand{\lbrak}[1]{\ensuremath{\left(#1\right.}}
\providecommand{\rbrak}[1]{\ensuremath{\left.#1\right)}}
\providecommand{\cbrak}[1]{\ensuremath{\left\{#1\right\}}}
\providecommand{\lcbrak}[1]{\ensuremath{\left\{#1\right.}}
\providecommand{\rcbrak}[1]{\ensuremath{\left.#1\right\}}}
\theoremstyle{remark}
\newtheorem{rem}{Remark}
\newcommand{\sgn}{\mathop{\mathrm{sgn}}}
\providecommand{\abs}[1]{\lvert#1\vert}
\providecommand{\res}[1]{\Res\displaylimits_{#1}} 
\providecommand{\norm}[1]{\lVert#1\rVert}
%\providecommand{\norm}[1]{\lVert#1\rVert}
\providecommand{\mtx}[1]{\mathbf{#1}}
\providecommand{\mean}[1]{E[ #1 ]}
\providecommand{\fourier}{\overset{\mathcal{F}}{ \rightleftharpoons}}
%\providecommand{\hilbert}{\overset{\mathcal{H}}{ \rightleftharpoons}}
\providecommand{\system}{\overset{\mathcal{H}}{ \longleftrightarrow}}
	%\newcommand{\solution}[2]{\textbf{Solution:}{#1}}
\newcommand{\solution}{\noindent \textbf{Solution: }}
\newcommand{\cosec}{\,\text{cosec}\,}
\providecommand{\dec}[2]{\ensuremath{\overset{#1}{\underset{#2}{\gtrless}}}}
\newcommand{\myvec}[1]{\ensuremath{\begin{pmatrix}#1\end{pmatrix}}}
\newcommand{\mydet}[1]{\ensuremath{\begin{vmatrix}#1\end{vmatrix}}}
\numberwithin{equation}{subsection}
\makeatletter
\@addtoreset{figure}{problem}
\makeatother
\let\StandardTheFigure\thefigure
\let\vec\mathbf
\renewcommand{\thefigure}{\theproblem}
\def\putbox#1#2#3{\makebox[0in][l]{\makebox[#1][l]{}\raisebox{\baselineskip}[0in][0in]{\raisebox{#2}[0in][0in]{#3}}}}
     \def\rightbox#1{\makebox[0in][r]{#1}}
     \def\centbox#1{\makebox[0in]{#1}}
     \def\topbox#1{\raisebox{-\baselineskip}[0in][0in]{#1}}
     \def\midbox#1{\raisebox{-0.5\baselineskip}[0in][0in]{#1}}
\vspace{3cm}
\title{ASSIGNMENT-8}
\author{Unnati Gupta}
\maketitle
\newpage
\bigskip
\renewcommand{\thefigure}{\theenumi}
\renewcommand{\thetable}{\theenumi}
Download all python codes from 
\begin{lstlisting}
https://github.com/unnatigupta2320/Assignment_8
\end{lstlisting}
%
and latex-tikz codes from 
%
\begin{lstlisting}
https://github.com/unnatigupta2320/Assignment_8
\end{lstlisting}
%
\section{Question No-2.22}
Give the magnitude and direction of the net force acting on a stone of mass 0.1 kg.
\begin{enumerate}[label=\alph*.)]
    \item Just after it is dropped from the window of a stationary train.
    \item Just after it is dropped from the window of a train running at a constant velocity of 36 km/h.
    \item Just after it is dropped from the window of a train accelerating with $1 ms^{-2}$
    \item Lying on the floor of a train which is accelerating with $1ms^{-2}$, the stone being at rest relative to the train.
\end{enumerate}
%
\section{Solution}
Given that:
\begin{align}
   \text{mass of stone, }m=0.1 kg
\end{align}
\begin{enumerate}[label=\alph*.)]
\item Here, the stone is just dropped from window of stationary train. 
\begin{itemize}
    \item So acceleration $\vec{a}$ will be equal to acceleration due to gravity $\vec{g}$.
\begin{align}
&\therefore \vec{a} =\vec{g}=10 ms^{-2} 
\\
&\implies \text{Net force, }\vec{F}=m\vec{a}
\\
&\implies \vec{F}=0.1\times10 \text{ N}
\\
&\implies \vec{F}=1\text{ N}
\end{align}
\item This force $\vec{F}$ will be \textbf{acting vertically downwards}.
\end{itemize}
\item Here velocity of train is constant.
\begin{align}
\therefore \text{acceleration, } \vec{a} =0.    
\end{align}
\begin{itemize}
\item No force acts on the stone due to motion of train.
\item The force $\vec{F}$ acting on stone will be weight of stone.
\begin{align}
&\therefore \vec{F} =\text{weight of stone }
\\
&\implies \vec{F}=m\vec{g}
\\
&\implies \vec{F}=0.1\times10 \text{ N}
\\
&\implies \vec{F}=1\text{ N}
\end{align}
\item This force $\vec{F}$ will also be \textbf{acting vertically downwards}.
\end{itemize}
\item When the train is accelerating with $1 ms^{-2}$ an additional force $\vec{F'}$ will be acting on stone where,
\begin{align}
 \vec{F'}&=m\vec{a}
 \\
 \vec{F'}&=0.1\times1\text{ N}
 \\
 \vec{F'}&=0.1\text{ N}
\end{align}
\begin{itemize}
\item This $\vec{F'}$ will be acting in the horizontal direction.But once the stone is dropped from the train, $\vec{F'}$ becomes zero.
\item Now,the force $\vec{F}$ acting on stone will be weight of stone.
\begin{align}
&\therefore \vec{F} =\text{weight of stone }
\\
&\implies \vec{F}=m\vec{g}
\\
&\implies \vec{F}=0.1\times10 \text{ N}
\\
&\implies \vec{F}=1\text{ N}
\end{align}
\item This force $\vec{F}$ will also be \textbf{acting vertically downwards}.
\end{itemize}
\item As the stone is lying on the floor of the train,its acceleration is the same as that of the train. 
\begin{itemize}
    \item So it's acceleration $\vec{a} = 1 ms^{-2}$.
\begin{align}
&\implies \text{Net force, }\vec{F}=m\vec{a}
\\
&\implies \vec{F}=0.1\times1 \text{ N}
\\
&\implies \vec{F}=0.1\text{ N}
\end{align}
\item This force is \textbf{along the horizontal direction of motion of the train}.
\end{itemize}
\end{enumerate}
\end{document}
